%%%%%%%%%%%%%%%%%%%%%%%%%%%%%%%%%%%%%%%%%
% University/School Laboratory Report
% LaTeX Template
% Version 3.0 (4/2/13)
%
% This template has been downloaded from:
% http://www.LaTeXTemplates.com
%
% Original author:
% Linux and Unix Users Group at Virginia Tech Wiki 
% (https://vtluug.org/wiki/Example_LaTeX_chem_lab_report)
%
% License:
% CC BY-NC-SA 3.0 (http://creativecommons.org/licenses/by-nc-sa/3.0/)
%
%%%%%%%%%%%%%%%%%%%%%%%%%%%%%%%%%%%%%%%%%

%----------------------------------------------------------------------------------------
%  PACKAGES AND DOCUMENT CONFIGURATIONS
%----------------------------------------------------------------------------------------

\documentclass{article}

\usepackage{siunitx} % Provides the \SI{}{} command for typesetting SI units

\usepackage{graphicx} % Required for the inclusion of images
\usepackage{caption}
\usepackage{subcaption}
\usepackage{float}
\usepackage{amsmath, amssymb}
\usepackage{mathtools}
\usepackage[usenames,dvipsnames,table]{xcolor}
\setlength\parindent{0pt} % Removes all indentation from paragraphs

%\renewcommand{\labelenumi}{\alph{enumi}.} % Make numbering in the enumerate environment by letter rather than number (e.g. section 6)

%\usepackage{times} % Uncomment to use the Times New Roman font

%----------------------------------------------------------------------------------------
%  DOCUMENT INFORMATION
%----------------------------------------------------------------------------------------

\title{Assignement 1 \\ High Performance Scientific Computing} % Title

\author{Pierre \textsc{Lorent} — Damien \textsc{Smeets}} % Author name

\date{\today} % Date for the report

\definecolor{light-gray}{gray}{0.95}
\definecolor{extra-gray}{gray}{0.75}
\definecolor{medium-gray}{gray}{0.85}

\begin{document}

\maketitle % Insert the title, author and date

\begin{center}
\begin{tabular}{l r}
Instructor: & Professor \textsc{Geuzaine}\\ % Instructor/supervisor
Assistant: & Amaury \textsc{Johnen}
\end{tabular}
\end{center}

\section{NAT}
\subsection{Generalities}
\subsubsection*{Exercice 1}
\begin{itemize}
\item Table making the translation between ports and hosts. You know what this is.
\item Difference between a static NAT and a dynamic NAT. Static are pre-configured, dynamic takes care of the mapping itself. Static allow servers to be behind NATs.
\item \begin{itemize}\item Advantages: security towards incoming attacks, public IP pool reduced, IAP can be easily changed, and internal network can be modified without notice.
\item Disadvantages: Modifies the packets, performance, no IPSEC without tunneling…\end{itemize}
\end{itemize}

%\begin{table}[H]
%\begin{center}
%\rowcolors{2}{light-gray}{medium-gray}
%\begin{tabular}{| c c c c | c c c c |}
%\hline
%\multicolumn{4}{| c |}{\cellcolor{extra-gray}Internal} & \multicolumn{4}{ c |}{\cellcolor{extra-gray}External} \\ \hline 
%\cellcolor{extra-gray} Source &\cellcolor{extra-gray} Port &\cellcolor{extra-gray} Dest. &\cellcolor{extra-gray} Port &\cellcolor{extra-gray} Source &\cellcolor{extra-gray} Port &\cellcolor{extra-gray} Dest. &\cellcolor{extra-gray} Port \\ \hline \hline \showrowcolors
%192.168.10.1 & 3001 & 128.178.52.93 & 80 & 193.49.96.60 & 3001 & 128.178.50.93 & 80 \\
%192.168.10.2 & 3001 & 128.178.52.93 & 80 & 193.49.96.61 & 3001 & 128.178.50.93 & 80 \\
%192.168.10.3 & 3001 & 128.178.52.93 & 80 & 193.49.96.62 & 3001 & 128.178.50.93 & 80 \\
%192.168.10.4 & 3001 & 128.178.52.93 & 80 & 193.49.96.60 & 3002 & 128.178.50.93 & 80 \\ \hline
%\end{tabular}
%\end{center}
%\end{table}

%\begin{figure}[H]
%  \centering
%  \includegraphics[width=1\textwidth]{Diag.png}
%\end{figure}

\end{document}
